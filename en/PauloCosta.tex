\documentclass[a4paper,10pt]{article}
\usepackage[top=2cm, bottom=2cm, left=2cm, right=2cm]{geometry}
\usepackage{helvet} \renewcommand{\familydefault}{\sfdefault}
\title{Curriculum - Paulo Costa}
\author{Paulo Costa}  
\usepackage[utf8]{inputenc}
\begin{document}
  \LARGE\textbf{Paulo Costa}

  \large me@paulo.costa.nom.br
  
  \large https://github.com/paulo-raca

%   \large Phone: +55 (19) 3241-3399

%   \large Campinas - SP - Brazil

  \normalsize 
  
  \section{Professional Experience}
    \begin{itemize}
      \item
        \textbf{Iperlane}, Consultant - since October/2016

        Clique para editar a descrição do cargoDevelopment of software containers for executing existing Android Apps.

      \item
        \textbf{Geofusion}, Software Engineer - April/2015 to October/2016
        
        My work consists on the development of the Java backend and Web frontend of Onmaps, a tool for analisys and visualization of geospatial data, used mostly as marketing tool (Geomarketing).

      \item
        \textbf{Facebook}, Software Engineer - October/2012 to September/2014
        
        I've worked on the project responsible for managing all of Facebook's servers and other datacenter devices, and which acts as an intermediary between several automation tools, such as: Provisioning, application deployment, repairs, financial systems, data replication, monitoring, etc.
        
        This project is written in Java and communicates with other applications through Thrift APIs. I also contributed with several client applications, usually written in Python, PHP or C++.

    
      \item  
        \textbf{Veridis Tecnologia}, Software Engineer - September/2009 to October/2012

        Veridis develops solutions for biometric access control (Fingerprints in special). I was responsible for developing the library for capture and  matching Fingerprints (Based on NBIS) and the firmware for a embedded system that controls biometric doors and turnstiles.
        
        Most Development was done in C++, but all libraries have APIs and samples in Java and C\#.

      \item  
        \textbf{Griaule Biometrics}, Software Engineer - May/2006 to September/2009

        Griaule sells a software library for capture and matching of fingerprints. I was responsible with supporting several fingerprint scanners, Linux support, modularization and other improvements.
        
        Most development was done in C++, but there are bindings and examples for Java, Delphi, C\# and VB.

      \item
        \textbf{CPqD - Research and Development Center in Telecomunications}, Intern - December/2005 to February/2006

        As an intern, my job consisted in fixing known bugs, writing JUnit tests and add new features to existing reusable components.

        All components are written in Java and tipically use Hibernate, EJB e Struts.
    \end{itemize}
    
  \section{Education}
    \begin{itemize}
      \item  
        \textbf{Computer Engineering} - IC/Unicamp - March/2006 to December/2010.

        Study Certificates:
        \begin{itemize}
          \item Visual Computing
          \item Project of Dedicated Hardware Systems
          \item Software Engineering
          \item Project of Information Systems
        \end{itemize}

        Grade Point Average: 7.7  (0-10)

%       \item  
%         \textbf{Computer Science} - IC/Unicamp - August/2011 to June/2013 (Expected)

      \item  
        \textbf{Technician in Eletronics} - Cotuca/Unicamp - February/2007 to December/2010

%         Grade Point Average: 7.3  (0-10)

      \item  
        \textbf{Technician in Informatics} - Cotuca/Unicamp - February/2003 to December/2006

        Emphasis in Support Systems

%         Grade Point Average: 8.8  (0-10)

    \end{itemize}
    
  \section{Languages}
    \begin{itemize}
      \item  
        \textbf{English} - Advanced Level.

        109/120 points on TOEFL iBT (November/2008).

      \item  
        \textbf{Portuguese} - Native Speaker.
    \end{itemize}


  \section{A few Projects}
    \begin{itemize}
      \item 
        \textbf{MongoFS} - FUSE filesystem to manage documents in Mongo databases.
    
      \item 
        \textbf{Contour Curves for Highcharts} - Plugin to the popular Highcharts library to plot 2-D and 3-D contour curves.
      
      \item 
        \textbf{Experimental Design} - Web interface to easily plan and analyze experiments using DOE methodologies.

        Made with Python/webapp2 and javascript.
        
      \item 
        \textbf{charlcd-gpio} e \textbf{ttyWiegand} - Linux kernel modules for driving HD44780-compatible character LCDs and to receive data from Wiegand deviced through GPIO pins.

      \item 
        \textbf{JIPS} - Java Virtual Machine, able to run any class file. Has partial support for JNI, no Garbage collection.

        Made with C++/Arch-C.

      \item
        \textbf{Space Wars} - StarWars-like 3-D space battle game.

        Made with Java, OpenGL, OpenAL and JInput.
        
      \item 
        \textbf{Mini-MIPS} - MIPS-like multi-cycle microprocessador. Implemented from behavioral VHDL to physical layout.

        Made with GHDL, Cadence Encounter and Cadence Virtuoso.

      \item 
        \textbf{Rabiscomatic} - Plotter assembled from old printers and a ARM microcontroller.

        Firmware and computer software written in C. USB interface.

      \item 
        \textbf{Calculatrix} - Plots math functions in in 2 or 3 dimensions.

        Made with Java and OpenGL.

      \item 
        \textbf{3D Wohoo} - Libraries for drawing 2-D and 3-D computer graphics, built from the scratch.

        Versions for Turbo Pascal, Delphi and Java.
    \end{itemize}

  \section{Programming Contests}
    \begin{itemize}
      \item 
        \textbf{ICPC – International Collegiate Programming Contest - World Finals}
        \begin{itemize}
          \item 2008 - Honorable mention - Team GAP/Unicamp
        \end{itemize}
      \item 
        \textbf{ICPC – International Collegiate Programming Contest - Brazilian Finals}
        \begin{itemize}
          \item 2007 (4$^{th}$ position - Team GAP/Unicamp), 2010 (8$^{th}$ position - Team Unicamp Alfa)
        \end{itemize}
      \item 
        \textbf{IOI - International Olympiads in Informatics}
        \begin{itemize}
          \item 2006 - 122$^{nd}$ position (Bronze)
        \end{itemize}
      \item 
        \textbf{CIIC - Ibero-american Contest in Informatics)}
        \begin{itemize}
          \item 2004 (Silver), 2006 (Silver)
        \end{itemize}
      \item
        \textbf{OBI – Brazilian Olympiads in Informatics}
        \begin{itemize}
          \item 2003 (Silver), 2004 (Golden), 2005 (Bronze), 2006 (Golden)
          \item 2010 and 2011 - Worked on the programming courses.
        \end{itemize}
      \item 
        \textbf{Google Code Jam Latin America}
        \begin{itemize}
          \item 2007 – 80$^{th}$ position
        \end{itemize}
    \end{itemize}

%   \section{Courses Taken}
%     \begin{itemize}
%       \item 
%         \textbf{OBI's Advanced Programming Course} - IC / Unicamp
%         \begin{itemize}
%           \item 2003, 2004 and 2006
%         \end{itemize}
%       \item 
%         \textbf{Summer Programming Challenges} - IME / USP
%         \begin{itemize}
%           \item 2010
%         \end{itemize}
%       \item 
%         \textbf{Compiler Transformations and Mapping Techniques for Reconfigurable Architectures} - ICMC / USP
%         \begin{itemize}
%           \item 2011
%         \end{itemize}
%     \end{itemize}


\end{document}
