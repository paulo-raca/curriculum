\documentclass[a4paper,10pt]{article}
\usepackage[top=2cm, bottom=2cm, left=2cm, right=2cm]{geometry}
\usepackage{helvet} \renewcommand{\familydefault}{\sfdefault}
\usepackage{hyperref}
\hypersetup{
    linkcolor=blue,
}

\title{Curriculum - Paulo Costa}
\author{Paulo Costa}  
\usepackage[utf8]{inputenc}
\begin{document}
  \LARGE\textbf{Paulo Costa}

  \large Software Engineer

  \large me@paulo.costa.nom.br
  
  \large https://github.com/paulo-raca

  % \large Phone: +55 (19) 98301-9923

  \large Campinas - SP - Brazil

  \normalsize 
  
  \section{Professional Experience}
    \begin{itemize}
      \item
        \textbf{Google} - 2024-06 to 2024-12

        Maintenance of large-scale data ingestion tools, used to import data from partners into Google Search.

      \item
        \textbf{SlashID} - 2023-01 to 2024-04

        Development of Identity Provider and Authorization Proxy in Go.

      \item
        \textbf{Pass2Go} - 2021-12 to 2023-01

        Development of Python + GraphQL backend for Pass2Go, a tool to speed up access to condos and office buildings with QR Codes.

      \item
        \textbf{Crowdstrike / Iperlane} - 2016-10 to 2021-10

        Development of software container for Android apps, capable of installing, monitoring and executing 3rd
        party apps isolated from the system and without extra permissions.
        
        The project was developed mostly in Java, with parts in C++, Kotlin and Python.

        I also worked on several modules and tools of interest, including:

        \begin{itemize}
          \item \textbf{ADB Proxy}: Uses SSH tunnels to attach the local computer to a remote Android device. Useful for remote control and debugging.
          \item \textbf{Jacoco Multiprocess}: Allows collecting coverage data on multiprocess Android apps
          \item \textbf{Android-Full-Framework} - Gradle plugin for Android projects that expose every method/field from every Android Framework version as public at compile-time and then modifies the bytecode with the necessary boilerplate to access it. Very handy for hacking Android's internals.
          \item \textbf{HiddenAPI} - Unlocks access to Android's non-SDK interfaces
        \end{itemize}

        I was initially hired by Iperlane, which was acquired by CrowdStrike in 2017-10.

      \item
        \textbf{Geofusion} - 2015-04 to 2016-10
        
        Development of the Java backend and Web frontend of Onmaps, a tool for analisys and visualization of geospatial data, used mostly as marketing tool (Geomarketing).
        
        The project uses a Java backend, Web frontend, Orable database and spatial indexes in Solr.

      \item
        \textbf{Facebook} - 2012-10 to 2014-09
        
        I've worked on the project responsible for managing all of Facebook's servers and other datacenter devices, and which acts as an intermediary between several automation tools, such as: Provisioning, application deployment, repairs, financial systems, data replication, monitoring, etc.
        
        This project is written in Java and communicates with other applications through Thrift APIs. I also contributed with several client applications, usually written in Python, PHP or C++.

    
      \item  
        \textbf{Veridis Tecnologia} - 2009-09 to 2012-10

        Veridis develops solutions for biometric access control (Fingerprints in special). I was responsible for developing the library for capture and  matching Fingerprints (Based on NBIS) and the firmware for a embedded system that controls biometric doors and turnstiles.
        
        Most Development was done in C++, but all libraries have APIs and samples in Java and C\#.

      \item  
        \textbf{Griaule Biometrics} - 2006-05 to 2009-09

        Griaule sells a software library for capture and matching of fingerprints. I was responsible with supporting several fingerprint scanners, Linux support, modularization and other improvements.
        
        Most development was done in C++, but there are bindings and examples for Java, Delphi, C\# and VB.

      \item
        \textbf{CPqD - Research and Development Center in Telecomunications}, Intern - 2005-12 to 2006-02

        As an intern, my job consisted in fixing known bugs, writing JUnit tests and add new features to existing components.

        All components are written in Java and tipically used Hibernate, EJB e Struts.
    \end{itemize}



  \section{Education}
    \begin{itemize}
      \item
        \textbf{Specialization in Automation and Control of Industrial and Agro-Industrial Processes} - Feagri/Unicamp - 2019-03 to 2020-03

        (Unfinished due to COVID lockdown)

      \item
        \textbf{Specialization in Complex Data Mining} - IC/Unicamp - 2018-03 to 2018-08.

      \item  
        \textbf{Computer Engineering} - IC/Unicamp - 2006-03 to 2010-12.

%         Grade Point Average: 7.7  (0-10)

        Study Certificates:
        \begin{itemize}
          \item Visual Computing
          \item Project of Dedicated Hardware Systems
          \item Software Engineering
          \item Project of Information Systems
        \end{itemize}

%       \item  
%         \textbf{Computer Science} - IC/Unicamp - 2011-08 to 2013-06 (Expected)

      \item  
        \textbf{Technician in Eletronics} - Cotuca/Unicamp - 2007-02 to 2010-12

%         Grade Point Average: 7.3  (0-10)

      \item  
        \textbf{Technician in Informatics} - Cotuca/Unicamp - 2003-02 to 2006-12

        Emphasis in Support Systems

%         Grade Point Average: 8.8  (0-10)

    \end{itemize}



  \section{Languages}
    \begin{itemize}

      \item
        \textbf{Portuguese} - Native Speaker.

      \item
        \textbf{English} - Advanced Level.

        109/120 points on TOEFL iBT (2008-11).

    \end{itemize}



  \section{Some personal Projects}
    \begin{itemize}

      \item
        \textbf{u2fdev} FIDO authenticator running on microcontrollers, Android and linux.

      \item
        \textbf{MongoFS} and \textbf{SpotifyFS}- FUSE filesystems to manage documents in Mongo databases as JSON files and play Spotify songs as MP3 files.

      \item
        \textbf{aionettools} Network-related tools written in asyncio Python. Currently supports ping for measuring latency/loss and NDT7 for measuring download/upload speed.

      \item
        \textbf{Contour Curves for Highcharts} - Plugin to the popular Highcharts library to plot 2-D and 3-D contour curves.

      \item
        \textbf{Experimental Design} - Web interface to easily plan and analyze experiments using DOE methodologies.

      \item
        \textbf{charlcd-gpio} e \textbf{ttyWiegand} - Linux kernel modules for driving HD44780-compatible character LCDs and to receive data from Wiegand deviced through GPIO pins.

      \item
        \textbf{JIPS} - Toy Java VM implemented in ArchC. Partial support for JNI, no Garbage collection, no threads.

      \item
        \textbf{Space Wars} - StarWars-like 3-D space battle game writen with Java + OpenGL + OpenAL.

      \item
        \textbf{Mini-MIPS} - A MIPS-like multi-cycle microprocessador, developed all the way from behavioral VHDL to physical layout. Used GHDL, Cadence Encounter and Cadence Virtuoso.

      \item
        \textbf{Rabiscomatic} - Plotter assembled from old printers and a ARM microcontroller.

      \item
        \textbf{Calculatrix} - Plots math functions in in 2 or 3 dimensions. Developed with Java and OpenGL.

      \item
        \textbf{3D Wohoo} - Libraries for drawing 2-D and 3-D computer graphics, built from the scratch in Pascal/Delphi and Java.
    \end{itemize}


%   \section{Programming Contests}
%     \begin{itemize}
%       \item
%         \textbf{ICPC – International Collegiate Programming Contest - World Finals}
%         \begin{itemize}
%           \item 2008 - Honorable mention - Team GAP/Unicamp
%         \end{itemize}
%       \item
%         \textbf{ICPC – International Collegiate Programming Contest - Brazilian Finals}
%         \begin{itemize}
%           \item 2007 (4$^{th}$ position - Team GAP/Unicamp), 2010 (8$^{th}$ position - Team Unicamp Alfa)
%         \end{itemize}
%       \item
%         \textbf{IOI - International Olympiads in Informatics}
%         \begin{itemize}
%           \item 2006 - 122$^{nd}$ position (Bronze)
%         \end{itemize}
%       \item
%         \textbf{CIIC - Ibero-american Contest in Informatics)}
%         \begin{itemize}
%           \item 2004 (Silver), 2006 (Silver)
%         \end{itemize}
%       \item
%         \textbf{OBI – Brazilian Olympiads in Informatics}
%         \begin{itemize}
%           \item 2003 (Silver), 2004 (Gold), 2005 (Bronze) and 2006 (Gold)
%           \item 2010 and 2011 - Helped on the programming courses.
%         \end{itemize}
%       \item
%         \textbf{Google Code Jam Latin America}
%         \begin{itemize}
%           \item 2007 – 80$^{th}$ position
%         \end{itemize}
%     \end{itemize}



%   \section{Courses Taken}
%     \begin{itemize}
%       \item
%         \textbf{OBI's Advanced Programming Course} - IC / Unicamp
%         \begin{itemize}
%           \item 2003, 2004 and 2006
%         \end{itemize}
%       \item
%         \textbf{Summer Programming Challenges} - IME / USP
%         \begin{itemize}
%           \item 2010
%         \end{itemize}
%       \item
%         \textbf{Compiler Transformations and Mapping Techniques for Reconfigurable Architectures} - ICMC / USP
%         \begin{itemize}
%           \item 2011
%         \end{itemize}
%     \end{itemize}

\end{document}
