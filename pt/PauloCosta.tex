\documentclass[a4paper,10pt]{article}
\usepackage[top=3cm, bottom=3cm, left=3cm, right=3cm]{geometry} 
\usepackage{helvet} \renewcommand{\familydefault}{\sfdefault}
\title{Currículo de Paulo Costa}
\author{Paulo Costa}  
\usepackage[utf8]{inputenc}
\begin{document}  
  \LARGE\textbf{Paulo Costa}

  \large eu@paulo.costa.nom.br

%   \large Telefone: (19) 3241-3399

%   \large Campinas - SP %- Brasil

  \normalsize 

  \section{Experiência Profissional}
    \begin{itemize}
      \item
        \textbf{Facebook} - Outubro/2012 a Setembro/2014
        
        Trabalhei no projeto responsável por gerenciar servidores e demais equipamentos de Datacenter, o qual atua como intermediário entre diversas ferramentas de automação: Provisionamento, distribuição de aplicações, reparos, sistemas financeiros, replicagem de dados, monitoramento, e muitos outros.
        
        Este projeto foi escrito em Java e se comunica com as demais aplicações via Thrift. Também Contribuí com diversas das aplicações clientes, as quais são normalmente são escritas em Python, PHP ou C++.

      \item
        \textbf{Veridis Tecnologia} - Setembro/2009 a Setembro/2012

        A Veridis desenvolve soluções de controle de acesso através de biometria (Especialmente impressões digitais). Fui responsável por desenvolver uma biblioteca para captura e comparação de impressões digitais (Baseada no NBIS), assim como o firmware de um sistema embarcado para controle de portas e catracas.
        
        A maior parte do desenvolvimento foi feita em C++, mas as bibliotecas também possuem APIs e exemplos de uso em Java e C\#.

      \item
        \textbf{Griaule Biometrics} - Maio/2006 a Setembro/2009
        
        A Griaule desenvolve uma biblioteca para captura e comparação de impressões digitais. Fui responsável por suportar diversos equipamentos de captura, suporte a Linux, modularização da biblioteca e outras melhorias.
        
        A maior parte do desenvolvimento era feita em C++, mas a biblioteca também possue APIs e exemplos de uso em Java, Delphi, C\# e VB.

      \item
        \textbf{CPqD - Centro de Pesquisa e Desenvolvimento em Telecomunicações} - Dezembro/2005 to Fevereiro/2006

        Como estagiário, meu trabalho consistia em corrigir bugs conhecidos, escrever testes JUnit e adicionar novas funcionalidades a componentes existentes.

        Todos os componentes eram escritos em Java e tipicamente utilizavam os frameworks Hibernate, EJB e Struts.
    \end{itemize}

  \section{Formação Acadêmica}
    \begin{itemize}
      \item  
        \textbf{Engenharia de Computação} - IC/Unicamp - Março/2006 a Dezembro/2010

        Certificados de Estudos em:
        \begin{itemize}
          \item Computação Visual
          \item Projeto de Sistemas de Hardware Dedicados
          \item Engenharia de Software
          \item Projeto de Sistemas de Informação
        \end{itemize}

%         Média: 7,7 (0-10)

%       \item  
%         \textbf{Ciência da Computação} - IC/Unicamp - Agosto/2011 a Junho/2013 (Esperado).

      \item  
        \textbf{Técnico em Eletroeletrônica} - Cotuca/Unicamp - Fevereiro/2007 a Dezembro/2010

%         Média: 7,3  (0-10)

      \item  
        \textbf{Técnico em Informática} - Cotuca/Unicamp - Fevereiro/2003 a Dezembro/2006

        Ênfase em Sistemas de Suporte

%         Média: 8,8  (0-10)

    \end{itemize}
%     
%   \section{Idiomas}
%     \begin{itemize}
%       \item  
%         \textbf{Inglês} - Nível avançado.
% 
%         109/120 pontos no TOEFL iBT (Novembro/2008).
% 
% %       \item  
% %         \textbf{Português} - Nativo.
%     \end{itemize}


  \section{Alguns Projetos}
    \begin{itemize}
      \item 
        \textbf{Curvas de nível para Highcharts} - Plugin para a biblioteca Highcharts para gerar curvas de nível em 2-D e 3-D.
      
      \item 
        \textbf{Experimental Design} - Interface web para análise de experimentos utilizando metodologias DOE.

        Feito com Python/webapp2 e javascript.
        
      \item 
        \textbf{charlcd-gpio} e \textbf{ttyWiegand} - Módulos para o Kernel Linux para o uso de LCDs de caracteres HD44780 conectados via GPIOs, e para comunicação com dispositivos Wiegand conectados via GPIOs.

      \item 
        \textbf{JIPS} - Máquina virtual Java simples. Possui suporte parcial a JNI, porém não suporta \emph{Garbage Collection}.

        Programado usando C++/Arch-C.

      \item 
        \textbf{Space Wars} - Jogo de batalha espacial 3D, estilo \emph{Star-Wars}.

        Programado usando Java, OpenGL, OpenAL e JInput.

      \item 
        \textbf{Mini-MIPS} - Projeto de um processador multi-ciclo semelhante ao MIPS. Foram implementadas todas as etapas, do VHDL comportamental ao layout físico.

        Feito com GHDL, Cadence Encounter e Cadence Virtuoso.

      \item 
        \textbf{Rabiscomático} - Plotter montado a partir de impressoras antigas e um microcontrolador.

        Firmware e software controlador escritos em C. Interface USB.

      \item 
        \textbf{Calculatrix} - Desenha gráficos de funções matemáticas em 2 ou 3 dimensões.

        Programado usando Java e OpenGL.

      \item 
        \textbf{3D Wohoo} - Bibliotecas de computação gráfica em 2 e 3 dimensões, implementados do zero.

        Versões em Turbo Pascal, Delphi e Java.
    \end{itemize}

  \section{Competições de programação}
    \begin{itemize}
      \item 
        \textbf{ICPC – International Collegiate Programming Contest - Final Mundial}
        \begin{itemize}
          \item 2008 (Menção honrosa - Equipe GAP/Unicamp)
        \end{itemize}
      \item 
        \textbf{Maratona Brasileira de Programação - Final Brasileira}
        \begin{itemize}
          \item 2007 (4º lugar - Equipe GAP/Unicamp), 2010 (8º lugar - Equipe Alpha/Unicamp)
        \end{itemize}
      \item 
        \textbf{IOI - International Olympiads in Informatics}
        \begin{itemize}
          \item 2006 - 122º lugar (Bronze)
        \end{itemize}
      \item 
        \textbf{CIIC - Competencia Iberoamericana de Informática por Correspondencia}
        \begin{itemize}
          \item 2004 (Prata), 2006 (Prata)
        \end{itemize}
      \item 
        \textbf{OBI – Olimpíada Brasileira de Informática}
        \begin{itemize}
          \item 2003 (Prata), 2004 (Ouro), 2005 (Bronze), 2006 (Ouro)
          \item 2010 e 2011 (Monitor dos cursos)
        \end{itemize}
      \item 
        \textbf{Google Code Jam Latin America}
        \begin{itemize}
          \item 2007 – 80º Colocado
        \end{itemize}
    \end{itemize}
% 
%   \section{Cursos}
%     \begin{itemize}
%       \item 
%         \textbf{Curso de Programação Avançada da OBI} - IC / Unicamp
%         \begin{itemize}
%           \item 2003, 2004 e 2006
%         \end{itemize}
%       \item 
%         \textbf{Desafios de Programação no Verão} - IME / USP
%         \begin{itemize}
%           \item 2010
%         \end{itemize}
%       \item 
%         \textbf{Compiler Transformations and Mapping Techniques for Reconfigurable Architectures} - ICMC / USP
%         \begin{itemize}
%           \item 2011
%         \end{itemize}
%     \end{itemize}


\end{document}
